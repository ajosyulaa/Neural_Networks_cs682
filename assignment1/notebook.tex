
% Default to the notebook output style

    


% Inherit from the specified cell style.




    
\documentclass[11pt]{article}

    
    
    \usepackage[T1]{fontenc}
    % Nicer default font (+ math font) than Computer Modern for most use cases
    \usepackage{mathpazo}

    % Basic figure setup, for now with no caption control since it's done
    % automatically by Pandoc (which extracts ![](path) syntax from Markdown).
    \usepackage{graphicx}
    % We will generate all images so they have a width \maxwidth. This means
    % that they will get their normal width if they fit onto the page, but
    % are scaled down if they would overflow the margins.
    \makeatletter
    \def\maxwidth{\ifdim\Gin@nat@width>\linewidth\linewidth
    \else\Gin@nat@width\fi}
    \makeatother
    \let\Oldincludegraphics\includegraphics
    % Set max figure width to be 80% of text width, for now hardcoded.
    \renewcommand{\includegraphics}[1]{\Oldincludegraphics[width=.8\maxwidth]{#1}}
    % Ensure that by default, figures have no caption (until we provide a
    % proper Figure object with a Caption API and a way to capture that
    % in the conversion process - todo).
    \usepackage{caption}
    \DeclareCaptionLabelFormat{nolabel}{}
    \captionsetup{labelformat=nolabel}

    \usepackage{adjustbox} % Used to constrain images to a maximum size 
    \usepackage{xcolor} % Allow colors to be defined
    \usepackage{enumerate} % Needed for markdown enumerations to work
    \usepackage{geometry} % Used to adjust the document margins
    \usepackage{amsmath} % Equations
    \usepackage{amssymb} % Equations
    \usepackage{textcomp} % defines textquotesingle
    % Hack from http://tex.stackexchange.com/a/47451/13684:
    \AtBeginDocument{%
        \def\PYZsq{\textquotesingle}% Upright quotes in Pygmentized code
    }
    \usepackage{upquote} % Upright quotes for verbatim code
    \usepackage{eurosym} % defines \euro
    \usepackage[mathletters]{ucs} % Extended unicode (utf-8) support
    \usepackage[utf8x]{inputenc} % Allow utf-8 characters in the tex document
    \usepackage{fancyvrb} % verbatim replacement that allows latex
    \usepackage{grffile} % extends the file name processing of package graphics 
                         % to support a larger range 
    % The hyperref package gives us a pdf with properly built
    % internal navigation ('pdf bookmarks' for the table of contents,
    % internal cross-reference links, web links for URLs, etc.)
    \usepackage{hyperref}
    \usepackage{longtable} % longtable support required by pandoc >1.10
    \usepackage{booktabs}  % table support for pandoc > 1.12.2
    \usepackage[inline]{enumitem} % IRkernel/repr support (it uses the enumerate* environment)
    \usepackage[normalem]{ulem} % ulem is needed to support strikethroughs (\sout)
                                % normalem makes italics be italics, not underlines
    

    
    
    % Colors for the hyperref package
    \definecolor{urlcolor}{rgb}{0,.145,.698}
    \definecolor{linkcolor}{rgb}{.71,0.21,0.01}
    \definecolor{citecolor}{rgb}{.12,.54,.11}

    % ANSI colors
    \definecolor{ansi-black}{HTML}{3E424D}
    \definecolor{ansi-black-intense}{HTML}{282C36}
    \definecolor{ansi-red}{HTML}{E75C58}
    \definecolor{ansi-red-intense}{HTML}{B22B31}
    \definecolor{ansi-green}{HTML}{00A250}
    \definecolor{ansi-green-intense}{HTML}{007427}
    \definecolor{ansi-yellow}{HTML}{DDB62B}
    \definecolor{ansi-yellow-intense}{HTML}{B27D12}
    \definecolor{ansi-blue}{HTML}{208FFB}
    \definecolor{ansi-blue-intense}{HTML}{0065CA}
    \definecolor{ansi-magenta}{HTML}{D160C4}
    \definecolor{ansi-magenta-intense}{HTML}{A03196}
    \definecolor{ansi-cyan}{HTML}{60C6C8}
    \definecolor{ansi-cyan-intense}{HTML}{258F8F}
    \definecolor{ansi-white}{HTML}{C5C1B4}
    \definecolor{ansi-white-intense}{HTML}{A1A6B2}

    % commands and environments needed by pandoc snippets
    % extracted from the output of `pandoc -s`
    \providecommand{\tightlist}{%
      \setlength{\itemsep}{0pt}\setlength{\parskip}{0pt}}
    \DefineVerbatimEnvironment{Highlighting}{Verbatim}{commandchars=\\\{\}}
    % Add ',fontsize=\small' for more characters per line
    \newenvironment{Shaded}{}{}
    \newcommand{\KeywordTok}[1]{\textcolor[rgb]{0.00,0.44,0.13}{\textbf{{#1}}}}
    \newcommand{\DataTypeTok}[1]{\textcolor[rgb]{0.56,0.13,0.00}{{#1}}}
    \newcommand{\DecValTok}[1]{\textcolor[rgb]{0.25,0.63,0.44}{{#1}}}
    \newcommand{\BaseNTok}[1]{\textcolor[rgb]{0.25,0.63,0.44}{{#1}}}
    \newcommand{\FloatTok}[1]{\textcolor[rgb]{0.25,0.63,0.44}{{#1}}}
    \newcommand{\CharTok}[1]{\textcolor[rgb]{0.25,0.44,0.63}{{#1}}}
    \newcommand{\StringTok}[1]{\textcolor[rgb]{0.25,0.44,0.63}{{#1}}}
    \newcommand{\CommentTok}[1]{\textcolor[rgb]{0.38,0.63,0.69}{\textit{{#1}}}}
    \newcommand{\OtherTok}[1]{\textcolor[rgb]{0.00,0.44,0.13}{{#1}}}
    \newcommand{\AlertTok}[1]{\textcolor[rgb]{1.00,0.00,0.00}{\textbf{{#1}}}}
    \newcommand{\FunctionTok}[1]{\textcolor[rgb]{0.02,0.16,0.49}{{#1}}}
    \newcommand{\RegionMarkerTok}[1]{{#1}}
    \newcommand{\ErrorTok}[1]{\textcolor[rgb]{1.00,0.00,0.00}{\textbf{{#1}}}}
    \newcommand{\NormalTok}[1]{{#1}}
    
    % Additional commands for more recent versions of Pandoc
    \newcommand{\ConstantTok}[1]{\textcolor[rgb]{0.53,0.00,0.00}{{#1}}}
    \newcommand{\SpecialCharTok}[1]{\textcolor[rgb]{0.25,0.44,0.63}{{#1}}}
    \newcommand{\VerbatimStringTok}[1]{\textcolor[rgb]{0.25,0.44,0.63}{{#1}}}
    \newcommand{\SpecialStringTok}[1]{\textcolor[rgb]{0.73,0.40,0.53}{{#1}}}
    \newcommand{\ImportTok}[1]{{#1}}
    \newcommand{\DocumentationTok}[1]{\textcolor[rgb]{0.73,0.13,0.13}{\textit{{#1}}}}
    \newcommand{\AnnotationTok}[1]{\textcolor[rgb]{0.38,0.63,0.69}{\textbf{\textit{{#1}}}}}
    \newcommand{\CommentVarTok}[1]{\textcolor[rgb]{0.38,0.63,0.69}{\textbf{\textit{{#1}}}}}
    \newcommand{\VariableTok}[1]{\textcolor[rgb]{0.10,0.09,0.49}{{#1}}}
    \newcommand{\ControlFlowTok}[1]{\textcolor[rgb]{0.00,0.44,0.13}{\textbf{{#1}}}}
    \newcommand{\OperatorTok}[1]{\textcolor[rgb]{0.40,0.40,0.40}{{#1}}}
    \newcommand{\BuiltInTok}[1]{{#1}}
    \newcommand{\ExtensionTok}[1]{{#1}}
    \newcommand{\PreprocessorTok}[1]{\textcolor[rgb]{0.74,0.48,0.00}{{#1}}}
    \newcommand{\AttributeTok}[1]{\textcolor[rgb]{0.49,0.56,0.16}{{#1}}}
    \newcommand{\InformationTok}[1]{\textcolor[rgb]{0.38,0.63,0.69}{\textbf{\textit{{#1}}}}}
    \newcommand{\WarningTok}[1]{\textcolor[rgb]{0.38,0.63,0.69}{\textbf{\textit{{#1}}}}}
    
    
    % Define a nice break command that doesn't care if a line doesn't already
    % exist.
    \def\br{\hspace*{\fill} \\* }
    % Math Jax compatability definitions
    \def\gt{>}
    \def\lt{<}
    % Document parameters
    \title{knn}
    
    
    

    % Pygments definitions
    
\makeatletter
\def\PY@reset{\let\PY@it=\relax \let\PY@bf=\relax%
    \let\PY@ul=\relax \let\PY@tc=\relax%
    \let\PY@bc=\relax \let\PY@ff=\relax}
\def\PY@tok#1{\csname PY@tok@#1\endcsname}
\def\PY@toks#1+{\ifx\relax#1\empty\else%
    \PY@tok{#1}\expandafter\PY@toks\fi}
\def\PY@do#1{\PY@bc{\PY@tc{\PY@ul{%
    \PY@it{\PY@bf{\PY@ff{#1}}}}}}}
\def\PY#1#2{\PY@reset\PY@toks#1+\relax+\PY@do{#2}}

\expandafter\def\csname PY@tok@w\endcsname{\def\PY@tc##1{\textcolor[rgb]{0.73,0.73,0.73}{##1}}}
\expandafter\def\csname PY@tok@c\endcsname{\let\PY@it=\textit\def\PY@tc##1{\textcolor[rgb]{0.25,0.50,0.50}{##1}}}
\expandafter\def\csname PY@tok@cp\endcsname{\def\PY@tc##1{\textcolor[rgb]{0.74,0.48,0.00}{##1}}}
\expandafter\def\csname PY@tok@k\endcsname{\let\PY@bf=\textbf\def\PY@tc##1{\textcolor[rgb]{0.00,0.50,0.00}{##1}}}
\expandafter\def\csname PY@tok@kp\endcsname{\def\PY@tc##1{\textcolor[rgb]{0.00,0.50,0.00}{##1}}}
\expandafter\def\csname PY@tok@kt\endcsname{\def\PY@tc##1{\textcolor[rgb]{0.69,0.00,0.25}{##1}}}
\expandafter\def\csname PY@tok@o\endcsname{\def\PY@tc##1{\textcolor[rgb]{0.40,0.40,0.40}{##1}}}
\expandafter\def\csname PY@tok@ow\endcsname{\let\PY@bf=\textbf\def\PY@tc##1{\textcolor[rgb]{0.67,0.13,1.00}{##1}}}
\expandafter\def\csname PY@tok@nb\endcsname{\def\PY@tc##1{\textcolor[rgb]{0.00,0.50,0.00}{##1}}}
\expandafter\def\csname PY@tok@nf\endcsname{\def\PY@tc##1{\textcolor[rgb]{0.00,0.00,1.00}{##1}}}
\expandafter\def\csname PY@tok@nc\endcsname{\let\PY@bf=\textbf\def\PY@tc##1{\textcolor[rgb]{0.00,0.00,1.00}{##1}}}
\expandafter\def\csname PY@tok@nn\endcsname{\let\PY@bf=\textbf\def\PY@tc##1{\textcolor[rgb]{0.00,0.00,1.00}{##1}}}
\expandafter\def\csname PY@tok@ne\endcsname{\let\PY@bf=\textbf\def\PY@tc##1{\textcolor[rgb]{0.82,0.25,0.23}{##1}}}
\expandafter\def\csname PY@tok@nv\endcsname{\def\PY@tc##1{\textcolor[rgb]{0.10,0.09,0.49}{##1}}}
\expandafter\def\csname PY@tok@no\endcsname{\def\PY@tc##1{\textcolor[rgb]{0.53,0.00,0.00}{##1}}}
\expandafter\def\csname PY@tok@nl\endcsname{\def\PY@tc##1{\textcolor[rgb]{0.63,0.63,0.00}{##1}}}
\expandafter\def\csname PY@tok@ni\endcsname{\let\PY@bf=\textbf\def\PY@tc##1{\textcolor[rgb]{0.60,0.60,0.60}{##1}}}
\expandafter\def\csname PY@tok@na\endcsname{\def\PY@tc##1{\textcolor[rgb]{0.49,0.56,0.16}{##1}}}
\expandafter\def\csname PY@tok@nt\endcsname{\let\PY@bf=\textbf\def\PY@tc##1{\textcolor[rgb]{0.00,0.50,0.00}{##1}}}
\expandafter\def\csname PY@tok@nd\endcsname{\def\PY@tc##1{\textcolor[rgb]{0.67,0.13,1.00}{##1}}}
\expandafter\def\csname PY@tok@s\endcsname{\def\PY@tc##1{\textcolor[rgb]{0.73,0.13,0.13}{##1}}}
\expandafter\def\csname PY@tok@sd\endcsname{\let\PY@it=\textit\def\PY@tc##1{\textcolor[rgb]{0.73,0.13,0.13}{##1}}}
\expandafter\def\csname PY@tok@si\endcsname{\let\PY@bf=\textbf\def\PY@tc##1{\textcolor[rgb]{0.73,0.40,0.53}{##1}}}
\expandafter\def\csname PY@tok@se\endcsname{\let\PY@bf=\textbf\def\PY@tc##1{\textcolor[rgb]{0.73,0.40,0.13}{##1}}}
\expandafter\def\csname PY@tok@sr\endcsname{\def\PY@tc##1{\textcolor[rgb]{0.73,0.40,0.53}{##1}}}
\expandafter\def\csname PY@tok@ss\endcsname{\def\PY@tc##1{\textcolor[rgb]{0.10,0.09,0.49}{##1}}}
\expandafter\def\csname PY@tok@sx\endcsname{\def\PY@tc##1{\textcolor[rgb]{0.00,0.50,0.00}{##1}}}
\expandafter\def\csname PY@tok@m\endcsname{\def\PY@tc##1{\textcolor[rgb]{0.40,0.40,0.40}{##1}}}
\expandafter\def\csname PY@tok@gh\endcsname{\let\PY@bf=\textbf\def\PY@tc##1{\textcolor[rgb]{0.00,0.00,0.50}{##1}}}
\expandafter\def\csname PY@tok@gu\endcsname{\let\PY@bf=\textbf\def\PY@tc##1{\textcolor[rgb]{0.50,0.00,0.50}{##1}}}
\expandafter\def\csname PY@tok@gd\endcsname{\def\PY@tc##1{\textcolor[rgb]{0.63,0.00,0.00}{##1}}}
\expandafter\def\csname PY@tok@gi\endcsname{\def\PY@tc##1{\textcolor[rgb]{0.00,0.63,0.00}{##1}}}
\expandafter\def\csname PY@tok@gr\endcsname{\def\PY@tc##1{\textcolor[rgb]{1.00,0.00,0.00}{##1}}}
\expandafter\def\csname PY@tok@ge\endcsname{\let\PY@it=\textit}
\expandafter\def\csname PY@tok@gs\endcsname{\let\PY@bf=\textbf}
\expandafter\def\csname PY@tok@gp\endcsname{\let\PY@bf=\textbf\def\PY@tc##1{\textcolor[rgb]{0.00,0.00,0.50}{##1}}}
\expandafter\def\csname PY@tok@go\endcsname{\def\PY@tc##1{\textcolor[rgb]{0.53,0.53,0.53}{##1}}}
\expandafter\def\csname PY@tok@gt\endcsname{\def\PY@tc##1{\textcolor[rgb]{0.00,0.27,0.87}{##1}}}
\expandafter\def\csname PY@tok@err\endcsname{\def\PY@bc##1{\setlength{\fboxsep}{0pt}\fcolorbox[rgb]{1.00,0.00,0.00}{1,1,1}{\strut ##1}}}
\expandafter\def\csname PY@tok@kc\endcsname{\let\PY@bf=\textbf\def\PY@tc##1{\textcolor[rgb]{0.00,0.50,0.00}{##1}}}
\expandafter\def\csname PY@tok@kd\endcsname{\let\PY@bf=\textbf\def\PY@tc##1{\textcolor[rgb]{0.00,0.50,0.00}{##1}}}
\expandafter\def\csname PY@tok@kn\endcsname{\let\PY@bf=\textbf\def\PY@tc##1{\textcolor[rgb]{0.00,0.50,0.00}{##1}}}
\expandafter\def\csname PY@tok@kr\endcsname{\let\PY@bf=\textbf\def\PY@tc##1{\textcolor[rgb]{0.00,0.50,0.00}{##1}}}
\expandafter\def\csname PY@tok@bp\endcsname{\def\PY@tc##1{\textcolor[rgb]{0.00,0.50,0.00}{##1}}}
\expandafter\def\csname PY@tok@fm\endcsname{\def\PY@tc##1{\textcolor[rgb]{0.00,0.00,1.00}{##1}}}
\expandafter\def\csname PY@tok@vc\endcsname{\def\PY@tc##1{\textcolor[rgb]{0.10,0.09,0.49}{##1}}}
\expandafter\def\csname PY@tok@vg\endcsname{\def\PY@tc##1{\textcolor[rgb]{0.10,0.09,0.49}{##1}}}
\expandafter\def\csname PY@tok@vi\endcsname{\def\PY@tc##1{\textcolor[rgb]{0.10,0.09,0.49}{##1}}}
\expandafter\def\csname PY@tok@vm\endcsname{\def\PY@tc##1{\textcolor[rgb]{0.10,0.09,0.49}{##1}}}
\expandafter\def\csname PY@tok@sa\endcsname{\def\PY@tc##1{\textcolor[rgb]{0.73,0.13,0.13}{##1}}}
\expandafter\def\csname PY@tok@sb\endcsname{\def\PY@tc##1{\textcolor[rgb]{0.73,0.13,0.13}{##1}}}
\expandafter\def\csname PY@tok@sc\endcsname{\def\PY@tc##1{\textcolor[rgb]{0.73,0.13,0.13}{##1}}}
\expandafter\def\csname PY@tok@dl\endcsname{\def\PY@tc##1{\textcolor[rgb]{0.73,0.13,0.13}{##1}}}
\expandafter\def\csname PY@tok@s2\endcsname{\def\PY@tc##1{\textcolor[rgb]{0.73,0.13,0.13}{##1}}}
\expandafter\def\csname PY@tok@sh\endcsname{\def\PY@tc##1{\textcolor[rgb]{0.73,0.13,0.13}{##1}}}
\expandafter\def\csname PY@tok@s1\endcsname{\def\PY@tc##1{\textcolor[rgb]{0.73,0.13,0.13}{##1}}}
\expandafter\def\csname PY@tok@mb\endcsname{\def\PY@tc##1{\textcolor[rgb]{0.40,0.40,0.40}{##1}}}
\expandafter\def\csname PY@tok@mf\endcsname{\def\PY@tc##1{\textcolor[rgb]{0.40,0.40,0.40}{##1}}}
\expandafter\def\csname PY@tok@mh\endcsname{\def\PY@tc##1{\textcolor[rgb]{0.40,0.40,0.40}{##1}}}
\expandafter\def\csname PY@tok@mi\endcsname{\def\PY@tc##1{\textcolor[rgb]{0.40,0.40,0.40}{##1}}}
\expandafter\def\csname PY@tok@il\endcsname{\def\PY@tc##1{\textcolor[rgb]{0.40,0.40,0.40}{##1}}}
\expandafter\def\csname PY@tok@mo\endcsname{\def\PY@tc##1{\textcolor[rgb]{0.40,0.40,0.40}{##1}}}
\expandafter\def\csname PY@tok@ch\endcsname{\let\PY@it=\textit\def\PY@tc##1{\textcolor[rgb]{0.25,0.50,0.50}{##1}}}
\expandafter\def\csname PY@tok@cm\endcsname{\let\PY@it=\textit\def\PY@tc##1{\textcolor[rgb]{0.25,0.50,0.50}{##1}}}
\expandafter\def\csname PY@tok@cpf\endcsname{\let\PY@it=\textit\def\PY@tc##1{\textcolor[rgb]{0.25,0.50,0.50}{##1}}}
\expandafter\def\csname PY@tok@c1\endcsname{\let\PY@it=\textit\def\PY@tc##1{\textcolor[rgb]{0.25,0.50,0.50}{##1}}}
\expandafter\def\csname PY@tok@cs\endcsname{\let\PY@it=\textit\def\PY@tc##1{\textcolor[rgb]{0.25,0.50,0.50}{##1}}}

\def\PYZbs{\char`\\}
\def\PYZus{\char`\_}
\def\PYZob{\char`\{}
\def\PYZcb{\char`\}}
\def\PYZca{\char`\^}
\def\PYZam{\char`\&}
\def\PYZlt{\char`\<}
\def\PYZgt{\char`\>}
\def\PYZsh{\char`\#}
\def\PYZpc{\char`\%}
\def\PYZdl{\char`\$}
\def\PYZhy{\char`\-}
\def\PYZsq{\char`\'}
\def\PYZdq{\char`\"}
\def\PYZti{\char`\~}
% for compatibility with earlier versions
\def\PYZat{@}
\def\PYZlb{[}
\def\PYZrb{]}
\makeatother


    % Exact colors from NB
    \definecolor{incolor}{rgb}{0.0, 0.0, 0.5}
    \definecolor{outcolor}{rgb}{0.545, 0.0, 0.0}



    
    % Prevent overflowing lines due to hard-to-break entities
    \sloppy 
    % Setup hyperref package
    \hypersetup{
      breaklinks=true,  % so long urls are correctly broken across lines
      colorlinks=true,
      urlcolor=urlcolor,
      linkcolor=linkcolor,
      citecolor=citecolor,
      }
    % Slightly bigger margins than the latex defaults
    
    \geometry{verbose,tmargin=1in,bmargin=1in,lmargin=1in,rmargin=1in}
    
    

    \begin{document}
    
    
    \maketitle
    
    

    
    \section{k-Nearest Neighbor (kNN)
exercise}\label{k-nearest-neighbor-knn-exercise}

\emph{Complete and hand in this completed worksheet (including its
outputs and any supporting code outside of the worksheet) with your
assignment submission. For more details see the
\href{https://compsci682-fa18.github.io/assignments2018/assignment1}{assignments
page} on the course website.}

The kNN classifier consists of two stages:

\begin{itemize}
\tightlist
\item
  During training, the classifier takes the training data and simply
  remembers it
\item
  During testing, kNN classifies every test image by comparing to all
  training images and transfering the labels of the k most similar
  training examples
\item
  The value of k is cross-validated
\end{itemize}

In this exercise you will implement these steps and understand the basic
Image Classification pipeline, cross-validation, and gain proficiency in
writing efficient, vectorized code.

    \begin{Verbatim}[commandchars=\\\{\}]
{\color{incolor}In [{\color{incolor}1}]:} \PY{c+c1}{\PYZsh{} Run some setup code for this notebook.}
        
        \PY{k+kn}{import} \PY{n+nn}{random}
        \PY{k+kn}{import} \PY{n+nn}{numpy} \PY{k}{as} \PY{n+nn}{np}
        \PY{k+kn}{from} \PY{n+nn}{cs682}\PY{n+nn}{.}\PY{n+nn}{data\PYZus{}utils} \PY{k}{import} \PY{n}{load\PYZus{}CIFAR10}
        \PY{k+kn}{import} \PY{n+nn}{matplotlib}\PY{n+nn}{.}\PY{n+nn}{pyplot} \PY{k}{as} \PY{n+nn}{plt}
        
        \PY{k+kn}{from} \PY{n+nn}{\PYZus{}\PYZus{}future\PYZus{}\PYZus{}} \PY{k}{import} \PY{n}{print\PYZus{}function}
        
        \PY{c+c1}{\PYZsh{} This is a bit of magic to make matplotlib figures appear inline in the notebook}
        \PY{c+c1}{\PYZsh{} rather than in a new window.}
        \PY{o}{\PYZpc{}}\PY{k}{matplotlib} inline
        \PY{n}{plt}\PY{o}{.}\PY{n}{rcParams}\PY{p}{[}\PY{l+s+s1}{\PYZsq{}}\PY{l+s+s1}{figure.figsize}\PY{l+s+s1}{\PYZsq{}}\PY{p}{]} \PY{o}{=} \PY{p}{(}\PY{l+m+mf}{10.0}\PY{p}{,} \PY{l+m+mf}{8.0}\PY{p}{)} \PY{c+c1}{\PYZsh{} set default size of plots}
        \PY{n}{plt}\PY{o}{.}\PY{n}{rcParams}\PY{p}{[}\PY{l+s+s1}{\PYZsq{}}\PY{l+s+s1}{image.interpolation}\PY{l+s+s1}{\PYZsq{}}\PY{p}{]} \PY{o}{=} \PY{l+s+s1}{\PYZsq{}}\PY{l+s+s1}{nearest}\PY{l+s+s1}{\PYZsq{}}
        \PY{n}{plt}\PY{o}{.}\PY{n}{rcParams}\PY{p}{[}\PY{l+s+s1}{\PYZsq{}}\PY{l+s+s1}{image.cmap}\PY{l+s+s1}{\PYZsq{}}\PY{p}{]} \PY{o}{=} \PY{l+s+s1}{\PYZsq{}}\PY{l+s+s1}{gray}\PY{l+s+s1}{\PYZsq{}}
        
        \PY{c+c1}{\PYZsh{} Some more magic so that the notebook will reload external python modules;}
        \PY{c+c1}{\PYZsh{} see http://stackoverflow.com/questions/1907993/autoreload\PYZhy{}of\PYZhy{}modules\PYZhy{}in\PYZhy{}ipython}
        \PY{o}{\PYZpc{}}\PY{k}{load\PYZus{}ext} autoreload
        \PY{o}{\PYZpc{}}\PY{k}{autoreload} 2
\end{Verbatim}


    \begin{Verbatim}[commandchars=\\\{\}]
{\color{incolor}In [{\color{incolor}2}]:} \PY{c+c1}{\PYZsh{} Load the raw CIFAR\PYZhy{}10 data.}
        \PY{n}{cifar10\PYZus{}dir} \PY{o}{=} \PY{l+s+s1}{\PYZsq{}}\PY{l+s+s1}{cs682/datasets/cifar\PYZhy{}10\PYZhy{}batches\PYZhy{}py}\PY{l+s+s1}{\PYZsq{}}
        
        \PY{c+c1}{\PYZsh{} Cleaning up variables to prevent loading data multiple times (which may cause memory issue)}
        \PY{k}{try}\PY{p}{:}
           \PY{k}{del} \PY{n}{X\PYZus{}train}\PY{p}{,} \PY{n}{y\PYZus{}train}
           \PY{k}{del} \PY{n}{X\PYZus{}test}\PY{p}{,} \PY{n}{y\PYZus{}test}
           \PY{n+nb}{print}\PY{p}{(}\PY{l+s+s1}{\PYZsq{}}\PY{l+s+s1}{Clear previously loaded data.}\PY{l+s+s1}{\PYZsq{}}\PY{p}{)}
        \PY{k}{except}\PY{p}{:}
           \PY{k}{pass}
        
        \PY{n}{X\PYZus{}train}\PY{p}{,} \PY{n}{y\PYZus{}train}\PY{p}{,} \PY{n}{X\PYZus{}test}\PY{p}{,} \PY{n}{y\PYZus{}test} \PY{o}{=} \PY{n}{load\PYZus{}CIFAR10}\PY{p}{(}\PY{n}{cifar10\PYZus{}dir}\PY{p}{)}
        
        \PY{c+c1}{\PYZsh{} As a sanity check, we print out the size of the training and test data.}
        \PY{n+nb}{print}\PY{p}{(}\PY{l+s+s1}{\PYZsq{}}\PY{l+s+s1}{Training data shape: }\PY{l+s+s1}{\PYZsq{}}\PY{p}{,} \PY{n}{X\PYZus{}train}\PY{o}{.}\PY{n}{shape}\PY{p}{)}
        \PY{n+nb}{print}\PY{p}{(}\PY{l+s+s1}{\PYZsq{}}\PY{l+s+s1}{Training labels shape: }\PY{l+s+s1}{\PYZsq{}}\PY{p}{,} \PY{n}{y\PYZus{}train}\PY{o}{.}\PY{n}{shape}\PY{p}{)}
        \PY{n+nb}{print}\PY{p}{(}\PY{l+s+s1}{\PYZsq{}}\PY{l+s+s1}{Test data shape: }\PY{l+s+s1}{\PYZsq{}}\PY{p}{,} \PY{n}{X\PYZus{}test}\PY{o}{.}\PY{n}{shape}\PY{p}{)}
        \PY{n+nb}{print}\PY{p}{(}\PY{l+s+s1}{\PYZsq{}}\PY{l+s+s1}{Test labels shape: }\PY{l+s+s1}{\PYZsq{}}\PY{p}{,} \PY{n}{y\PYZus{}test}\PY{o}{.}\PY{n}{shape}\PY{p}{)}
\end{Verbatim}


    \begin{Verbatim}[commandchars=\\\{\}]
Training data shape:  (50000, 32, 32, 3)
Training labels shape:  (50000,)
Test data shape:  (10000, 32, 32, 3)
Test labels shape:  (10000,)

    \end{Verbatim}

    \begin{Verbatim}[commandchars=\\\{\}]
{\color{incolor}In [{\color{incolor}3}]:} \PY{c+c1}{\PYZsh{} Visualize some examples from the dataset.}
        \PY{c+c1}{\PYZsh{} We show a few examples of training images from each class.}
        \PY{n}{classes} \PY{o}{=} \PY{p}{[}\PY{l+s+s1}{\PYZsq{}}\PY{l+s+s1}{plane}\PY{l+s+s1}{\PYZsq{}}\PY{p}{,} \PY{l+s+s1}{\PYZsq{}}\PY{l+s+s1}{car}\PY{l+s+s1}{\PYZsq{}}\PY{p}{,} \PY{l+s+s1}{\PYZsq{}}\PY{l+s+s1}{bird}\PY{l+s+s1}{\PYZsq{}}\PY{p}{,} \PY{l+s+s1}{\PYZsq{}}\PY{l+s+s1}{cat}\PY{l+s+s1}{\PYZsq{}}\PY{p}{,} \PY{l+s+s1}{\PYZsq{}}\PY{l+s+s1}{deer}\PY{l+s+s1}{\PYZsq{}}\PY{p}{,} \PY{l+s+s1}{\PYZsq{}}\PY{l+s+s1}{dog}\PY{l+s+s1}{\PYZsq{}}\PY{p}{,} \PY{l+s+s1}{\PYZsq{}}\PY{l+s+s1}{frog}\PY{l+s+s1}{\PYZsq{}}\PY{p}{,} \PY{l+s+s1}{\PYZsq{}}\PY{l+s+s1}{horse}\PY{l+s+s1}{\PYZsq{}}\PY{p}{,} \PY{l+s+s1}{\PYZsq{}}\PY{l+s+s1}{ship}\PY{l+s+s1}{\PYZsq{}}\PY{p}{,} \PY{l+s+s1}{\PYZsq{}}\PY{l+s+s1}{truck}\PY{l+s+s1}{\PYZsq{}}\PY{p}{]}
        \PY{n}{num\PYZus{}classes} \PY{o}{=} \PY{n+nb}{len}\PY{p}{(}\PY{n}{classes}\PY{p}{)}
        \PY{n}{samples\PYZus{}per\PYZus{}class} \PY{o}{=} \PY{l+m+mi}{7}
        \PY{k}{for} \PY{n}{y}\PY{p}{,} \PY{n+nb+bp}{cls} \PY{o+ow}{in} \PY{n+nb}{enumerate}\PY{p}{(}\PY{n}{classes}\PY{p}{)}\PY{p}{:}
            \PY{n}{idxs} \PY{o}{=} \PY{n}{np}\PY{o}{.}\PY{n}{flatnonzero}\PY{p}{(}\PY{n}{y\PYZus{}train} \PY{o}{==} \PY{n}{y}\PY{p}{)}
            \PY{n}{idxs} \PY{o}{=} \PY{n}{np}\PY{o}{.}\PY{n}{random}\PY{o}{.}\PY{n}{choice}\PY{p}{(}\PY{n}{idxs}\PY{p}{,} \PY{n}{samples\PYZus{}per\PYZus{}class}\PY{p}{,} \PY{n}{replace}\PY{o}{=}\PY{k+kc}{False}\PY{p}{)}
            \PY{k}{for} \PY{n}{i}\PY{p}{,} \PY{n}{idx} \PY{o+ow}{in} \PY{n+nb}{enumerate}\PY{p}{(}\PY{n}{idxs}\PY{p}{)}\PY{p}{:}
                \PY{n}{plt\PYZus{}idx} \PY{o}{=} \PY{n}{i} \PY{o}{*} \PY{n}{num\PYZus{}classes} \PY{o}{+} \PY{n}{y} \PY{o}{+} \PY{l+m+mi}{1}
                \PY{n}{plt}\PY{o}{.}\PY{n}{subplot}\PY{p}{(}\PY{n}{samples\PYZus{}per\PYZus{}class}\PY{p}{,} \PY{n}{num\PYZus{}classes}\PY{p}{,} \PY{n}{plt\PYZus{}idx}\PY{p}{)}
                \PY{n}{plt}\PY{o}{.}\PY{n}{imshow}\PY{p}{(}\PY{n}{X\PYZus{}train}\PY{p}{[}\PY{n}{idx}\PY{p}{]}\PY{o}{.}\PY{n}{astype}\PY{p}{(}\PY{l+s+s1}{\PYZsq{}}\PY{l+s+s1}{uint8}\PY{l+s+s1}{\PYZsq{}}\PY{p}{)}\PY{p}{)}
                \PY{n}{plt}\PY{o}{.}\PY{n}{axis}\PY{p}{(}\PY{l+s+s1}{\PYZsq{}}\PY{l+s+s1}{off}\PY{l+s+s1}{\PYZsq{}}\PY{p}{)}
                \PY{k}{if} \PY{n}{i} \PY{o}{==} \PY{l+m+mi}{0}\PY{p}{:}
                    \PY{n}{plt}\PY{o}{.}\PY{n}{title}\PY{p}{(}\PY{n+nb+bp}{cls}\PY{p}{)}
        \PY{n}{plt}\PY{o}{.}\PY{n}{show}\PY{p}{(}\PY{p}{)}
\end{Verbatim}


    \begin{center}
    \adjustimage{max size={0.9\linewidth}{0.9\paperheight}}{output_3_0.png}
    \end{center}
    { \hspace*{\fill} \\}
    
    \begin{Verbatim}[commandchars=\\\{\}]
{\color{incolor}In [{\color{incolor}4}]:} \PY{c+c1}{\PYZsh{} Subsample the data for more efficient code execution in this exercise}
        \PY{n}{num\PYZus{}training} \PY{o}{=} \PY{l+m+mi}{5000}
        \PY{n}{mask} \PY{o}{=} \PY{n+nb}{list}\PY{p}{(}\PY{n+nb}{range}\PY{p}{(}\PY{n}{num\PYZus{}training}\PY{p}{)}\PY{p}{)}
        \PY{n}{X\PYZus{}train} \PY{o}{=} \PY{n}{X\PYZus{}train}\PY{p}{[}\PY{n}{mask}\PY{p}{]}
        \PY{n}{y\PYZus{}train} \PY{o}{=} \PY{n}{y\PYZus{}train}\PY{p}{[}\PY{n}{mask}\PY{p}{]}
        
        \PY{n}{num\PYZus{}test} \PY{o}{=} \PY{l+m+mi}{500}
        \PY{n}{mask} \PY{o}{=} \PY{n+nb}{list}\PY{p}{(}\PY{n+nb}{range}\PY{p}{(}\PY{n}{num\PYZus{}test}\PY{p}{)}\PY{p}{)}
        \PY{n}{X\PYZus{}test} \PY{o}{=} \PY{n}{X\PYZus{}test}\PY{p}{[}\PY{n}{mask}\PY{p}{]}
        \PY{n}{y\PYZus{}test} \PY{o}{=} \PY{n}{y\PYZus{}test}\PY{p}{[}\PY{n}{mask}\PY{p}{]}
\end{Verbatim}


    \begin{Verbatim}[commandchars=\\\{\}]
{\color{incolor}In [{\color{incolor}5}]:} \PY{c+c1}{\PYZsh{} Reshape the image data into rows}
        \PY{n}{X\PYZus{}train} \PY{o}{=} \PY{n}{np}\PY{o}{.}\PY{n}{reshape}\PY{p}{(}\PY{n}{X\PYZus{}train}\PY{p}{,} \PY{p}{(}\PY{n}{X\PYZus{}train}\PY{o}{.}\PY{n}{shape}\PY{p}{[}\PY{l+m+mi}{0}\PY{p}{]}\PY{p}{,} \PY{o}{\PYZhy{}}\PY{l+m+mi}{1}\PY{p}{)}\PY{p}{)}
        \PY{n}{X\PYZus{}test} \PY{o}{=} \PY{n}{np}\PY{o}{.}\PY{n}{reshape}\PY{p}{(}\PY{n}{X\PYZus{}test}\PY{p}{,} \PY{p}{(}\PY{n}{X\PYZus{}test}\PY{o}{.}\PY{n}{shape}\PY{p}{[}\PY{l+m+mi}{0}\PY{p}{]}\PY{p}{,} \PY{o}{\PYZhy{}}\PY{l+m+mi}{1}\PY{p}{)}\PY{p}{)}
        \PY{n+nb}{print}\PY{p}{(}\PY{n}{X\PYZus{}train}\PY{o}{.}\PY{n}{shape}\PY{p}{,} \PY{n}{X\PYZus{}test}\PY{o}{.}\PY{n}{shape}\PY{p}{)}
\end{Verbatim}


    \begin{Verbatim}[commandchars=\\\{\}]
(5000, 3072) (500, 3072)

    \end{Verbatim}

    \begin{Verbatim}[commandchars=\\\{\}]
{\color{incolor}In [{\color{incolor}6}]:} \PY{k+kn}{from} \PY{n+nn}{cs682}\PY{n+nn}{.}\PY{n+nn}{classifiers} \PY{k}{import} \PY{n}{KNearestNeighbor}
        
        \PY{c+c1}{\PYZsh{} Create a kNN classifier instance. }
        \PY{c+c1}{\PYZsh{} Remember that training a kNN classifier is a noop: }
        \PY{c+c1}{\PYZsh{} the Classifier simply remembers the data and does no further processing }
        \PY{n}{classifier} \PY{o}{=} \PY{n}{KNearestNeighbor}\PY{p}{(}\PY{p}{)}
        \PY{n}{classifier}\PY{o}{.}\PY{n}{train}\PY{p}{(}\PY{n}{X\PYZus{}train}\PY{p}{,} \PY{n}{y\PYZus{}train}\PY{p}{)}
\end{Verbatim}


    We would now like to classify the test data with the kNN classifier.
Recall that we can break down this process into two steps:

\begin{enumerate}
\def\labelenumi{\arabic{enumi}.}
\tightlist
\item
  First we must compute the distances between all test examples and all
  train examples.
\item
  Given these distances, for each test example we find the k nearest
  examples and have them vote for the label
\end{enumerate}

Lets begin with computing the distance matrix between all training and
test examples. For example, if there are \textbf{Ntr} training examples
and \textbf{Nte} test examples, this stage should result in a
\textbf{Nte x Ntr} matrix where each element (i,j) is the distance
between the i-th test and j-th train example.

First, open \texttt{cs682/classifiers/k\_nearest\_neighbor.py} and
implement the function \texttt{compute\_distances\_two\_loops} that uses
a (very inefficient) double loop over all pairs of (test, train)
examples and computes the distance matrix one element at a time.

    \begin{Verbatim}[commandchars=\\\{\}]
{\color{incolor}In [{\color{incolor}7}]:} \PY{c+c1}{\PYZsh{} Open cs682/classifiers/k\PYZus{}nearest\PYZus{}neighbor.py and implement}
        \PY{c+c1}{\PYZsh{} compute\PYZus{}distances\PYZus{}two\PYZus{}loops.}
        \PY{c+c1}{\PYZsh{}print(X\PYZus{}train)}
        \PY{c+c1}{\PYZsh{} Test your implementation:}
        \PY{n}{dists} \PY{o}{=} \PY{n}{classifier}\PY{o}{.}\PY{n}{compute\PYZus{}distances\PYZus{}two\PYZus{}loops}\PY{p}{(}\PY{n}{X\PYZus{}test}\PY{p}{)}
        \PY{n+nb}{print}\PY{p}{(}\PY{n}{dists}\PY{o}{.}\PY{n}{shape}\PY{p}{)}
\end{Verbatim}


    \begin{Verbatim}[commandchars=\\\{\}]
(500, 5000)

    \end{Verbatim}

    \begin{Verbatim}[commandchars=\\\{\}]
{\color{incolor}In [{\color{incolor}8}]:} \PY{c+c1}{\PYZsh{} We can visualize the distance matrix: each row is a single test example and}
        \PY{c+c1}{\PYZsh{} its distances to training examples}
        \PY{n}{plt}\PY{o}{.}\PY{n}{imshow}\PY{p}{(}\PY{n}{dists}\PY{p}{,} \PY{n}{interpolation}\PY{o}{=}\PY{l+s+s1}{\PYZsq{}}\PY{l+s+s1}{none}\PY{l+s+s1}{\PYZsq{}}\PY{p}{)}
        \PY{n}{plt}\PY{o}{.}\PY{n}{show}\PY{p}{(}\PY{p}{)}
\end{Verbatim}


    \begin{center}
    \adjustimage{max size={0.9\linewidth}{0.9\paperheight}}{output_9_0.png}
    \end{center}
    { \hspace*{\fill} \\}
    
    \textbf{Inline Question \#1:} Notice the structured patterns in the
distance matrix, where some rows or columns are visible brighter. (Note
that with the default color scheme black indicates low distances while
white indicates high distances.)

\begin{itemize}
\tightlist
\item
  What in the data is the cause behind the distinctly bright rows?
\item
  What causes the columns?
\end{itemize}

    \textbf{Your Answer}: The data that causes the distinctly bright rows
are the data points in the train data that are far away from the test
data. The columns are caused by data points on the test data that are
far from the train data

    \begin{Verbatim}[commandchars=\\\{\}]
{\color{incolor}In [{\color{incolor}9}]:} \PY{c+c1}{\PYZsh{} Now implement the function predict\PYZus{}labels and run the code below:}
        \PY{c+c1}{\PYZsh{} We use k = 1 (which is Nearest Neighbor).}
        \PY{n}{y\PYZus{}test\PYZus{}pred} \PY{o}{=} \PY{n}{classifier}\PY{o}{.}\PY{n}{predict\PYZus{}labels}\PY{p}{(}\PY{n}{dists}\PY{p}{,} \PY{n}{k}\PY{o}{=}\PY{l+m+mi}{1}\PY{p}{)}
        
        \PY{c+c1}{\PYZsh{} Compute and print the fraction of correctly predicted examples}
        \PY{n}{num\PYZus{}correct} \PY{o}{=} \PY{n}{np}\PY{o}{.}\PY{n}{sum}\PY{p}{(}\PY{n}{y\PYZus{}test\PYZus{}pred} \PY{o}{==} \PY{n}{y\PYZus{}test}\PY{p}{)}
        \PY{n}{accuracy} \PY{o}{=} \PY{n+nb}{float}\PY{p}{(}\PY{n}{num\PYZus{}correct}\PY{p}{)} \PY{o}{/} \PY{n}{num\PYZus{}test}
        \PY{n+nb}{print}\PY{p}{(}\PY{l+s+s1}{\PYZsq{}}\PY{l+s+s1}{Got }\PY{l+s+si}{\PYZpc{}d}\PY{l+s+s1}{ / }\PY{l+s+si}{\PYZpc{}d}\PY{l+s+s1}{ correct =\PYZgt{} accuracy: }\PY{l+s+si}{\PYZpc{}f}\PY{l+s+s1}{\PYZsq{}} \PY{o}{\PYZpc{}} \PY{p}{(}\PY{n}{num\PYZus{}correct}\PY{p}{,} \PY{n}{num\PYZus{}test}\PY{p}{,} \PY{n}{accuracy}\PY{p}{)}\PY{p}{)}
\end{Verbatim}


    \begin{Verbatim}[commandchars=\\\{\}]
Got 137 / 500 correct => accuracy: 0.274000

    \end{Verbatim}

    You should expect to see approximately \texttt{27\%} accuracy. Now lets
try out a larger \texttt{k}, say \texttt{k\ =\ 5}:

    \begin{Verbatim}[commandchars=\\\{\}]
{\color{incolor}In [{\color{incolor}10}]:} \PY{n}{y\PYZus{}test\PYZus{}pred} \PY{o}{=} \PY{n}{classifier}\PY{o}{.}\PY{n}{predict\PYZus{}labels}\PY{p}{(}\PY{n}{dists}\PY{p}{,} \PY{n}{k}\PY{o}{=}\PY{l+m+mi}{5}\PY{p}{)}
         \PY{n}{num\PYZus{}correct} \PY{o}{=} \PY{n}{np}\PY{o}{.}\PY{n}{sum}\PY{p}{(}\PY{n}{y\PYZus{}test\PYZus{}pred} \PY{o}{==} \PY{n}{y\PYZus{}test}\PY{p}{)}
         \PY{n}{accuracy} \PY{o}{=} \PY{n+nb}{float}\PY{p}{(}\PY{n}{num\PYZus{}correct}\PY{p}{)} \PY{o}{/} \PY{n}{num\PYZus{}test}
         \PY{n+nb}{print}\PY{p}{(}\PY{l+s+s1}{\PYZsq{}}\PY{l+s+s1}{Got }\PY{l+s+si}{\PYZpc{}d}\PY{l+s+s1}{ / }\PY{l+s+si}{\PYZpc{}d}\PY{l+s+s1}{ correct =\PYZgt{} accuracy: }\PY{l+s+si}{\PYZpc{}f}\PY{l+s+s1}{\PYZsq{}} \PY{o}{\PYZpc{}} \PY{p}{(}\PY{n}{num\PYZus{}correct}\PY{p}{,} \PY{n}{num\PYZus{}test}\PY{p}{,} \PY{n}{accuracy}\PY{p}{)}\PY{p}{)}
\end{Verbatim}


    \begin{Verbatim}[commandchars=\\\{\}]
Got 139 / 500 correct => accuracy: 0.278000

    \end{Verbatim}

    You should expect to see a slightly better performance than with
\texttt{k\ =\ 1}.

    \textbf{Inline Question 2} We can also other distance metrics such as L1
distance. The performance of a Nearest Neighbor classifier that uses L1
distance will not change if (Select all that apply.): 1. The data is
preprocessed by subtracting the mean. 2. The data is preprocessed by
subtracting the mean and dividing by the standard deviation. 3. The
coordinate axes for the data are rotated. 4. None of the above.

\emph{Your Answer}: The data is preprocessed by subtracting the mean and
dividing by the standard deviation

\emph{Your explanation}: This is because l1 distance is by subtacting
the distance between train and test data. hence, when the data is
preprocessed, its posriton will change and hence the mean corresponding
to them will also change. By dividing it by the standard devation, you
are scaling down, hence it would effectively be the same.

    \begin{Verbatim}[commandchars=\\\{\}]
{\color{incolor}In [{\color{incolor}11}]:} \PY{c+c1}{\PYZsh{} Now lets speed up distance matrix computation by using partial vectorization}
         \PY{c+c1}{\PYZsh{} with one loop. Implement the function compute\PYZus{}distances\PYZus{}one\PYZus{}loop and run the}
         \PY{c+c1}{\PYZsh{} code below:}
         \PY{n}{dists\PYZus{}one} \PY{o}{=} \PY{n}{classifier}\PY{o}{.}\PY{n}{compute\PYZus{}distances\PYZus{}one\PYZus{}loop}\PY{p}{(}\PY{n}{X\PYZus{}test}\PY{p}{)}
         
         \PY{c+c1}{\PYZsh{} To ensure that our vectorized implementation is correct, we make sure that it}
         \PY{c+c1}{\PYZsh{} agrees with the naive implementation. There are many ways to decide whether}
         \PY{c+c1}{\PYZsh{} two matrices are similar; one of the simplest is the Frobenius norm. In case}
         \PY{c+c1}{\PYZsh{} you haven\PYZsq{}t seen it before, the Frobenius norm of two matrices is the square}
         \PY{c+c1}{\PYZsh{} root of the squared sum of differences of all elements; in other words, reshape}
         \PY{c+c1}{\PYZsh{} the matrices into vectors and compute the Euclidean distance between them.}
         \PY{n}{difference} \PY{o}{=} \PY{n}{np}\PY{o}{.}\PY{n}{linalg}\PY{o}{.}\PY{n}{norm}\PY{p}{(}\PY{n}{dists} \PY{o}{\PYZhy{}} \PY{n}{dists\PYZus{}one}\PY{p}{,} \PY{n+nb}{ord}\PY{o}{=}\PY{l+s+s1}{\PYZsq{}}\PY{l+s+s1}{fro}\PY{l+s+s1}{\PYZsq{}}\PY{p}{)}
         \PY{n+nb}{print}\PY{p}{(}\PY{l+s+s1}{\PYZsq{}}\PY{l+s+s1}{Difference was: }\PY{l+s+si}{\PYZpc{}f}\PY{l+s+s1}{\PYZsq{}} \PY{o}{\PYZpc{}} \PY{p}{(}\PY{n}{difference}\PY{p}{,} \PY{p}{)}\PY{p}{)}
         \PY{k}{if} \PY{n}{difference} \PY{o}{\PYZlt{}} \PY{l+m+mf}{0.001}\PY{p}{:}
             \PY{n+nb}{print}\PY{p}{(}\PY{l+s+s1}{\PYZsq{}}\PY{l+s+s1}{Good! The distance matrices are the same}\PY{l+s+s1}{\PYZsq{}}\PY{p}{)}
         \PY{k}{else}\PY{p}{:}
             \PY{n+nb}{print}\PY{p}{(}\PY{l+s+s1}{\PYZsq{}}\PY{l+s+s1}{Uh\PYZhy{}oh! The distance matrices are different}\PY{l+s+s1}{\PYZsq{}}\PY{p}{)}
\end{Verbatim}


    \begin{Verbatim}[commandchars=\\\{\}]
Difference was: 0.000000
Good! The distance matrices are the same

    \end{Verbatim}

    \begin{Verbatim}[commandchars=\\\{\}]
{\color{incolor}In [{\color{incolor}12}]:} \PY{c+c1}{\PYZsh{} Now implement the fully vectorized version inside compute\PYZus{}distances\PYZus{}no\PYZus{}loops}
         \PY{c+c1}{\PYZsh{} and run the code}
         \PY{n}{dists\PYZus{}two} \PY{o}{=} \PY{n}{classifier}\PY{o}{.}\PY{n}{compute\PYZus{}distances\PYZus{}no\PYZus{}loops}\PY{p}{(}\PY{n}{X\PYZus{}test}\PY{p}{)}
         
         \PY{c+c1}{\PYZsh{} check that the distance matrix agrees with the one we computed before:}
         \PY{n}{difference} \PY{o}{=} \PY{n}{np}\PY{o}{.}\PY{n}{linalg}\PY{o}{.}\PY{n}{norm}\PY{p}{(}\PY{n}{dists} \PY{o}{\PYZhy{}} \PY{n}{dists\PYZus{}two}\PY{p}{,} \PY{n+nb}{ord}\PY{o}{=}\PY{l+s+s1}{\PYZsq{}}\PY{l+s+s1}{fro}\PY{l+s+s1}{\PYZsq{}}\PY{p}{)}
         \PY{n+nb}{print}\PY{p}{(}\PY{l+s+s1}{\PYZsq{}}\PY{l+s+s1}{Difference was: }\PY{l+s+si}{\PYZpc{}f}\PY{l+s+s1}{\PYZsq{}} \PY{o}{\PYZpc{}} \PY{p}{(}\PY{n}{difference}\PY{p}{,} \PY{p}{)}\PY{p}{)}
         \PY{k}{if} \PY{n}{difference} \PY{o}{\PYZlt{}} \PY{l+m+mf}{0.001}\PY{p}{:}
             \PY{n+nb}{print}\PY{p}{(}\PY{l+s+s1}{\PYZsq{}}\PY{l+s+s1}{Good! The distance matrices are the same}\PY{l+s+s1}{\PYZsq{}}\PY{p}{)}
         \PY{k}{else}\PY{p}{:}
             \PY{n+nb}{print}\PY{p}{(}\PY{l+s+s1}{\PYZsq{}}\PY{l+s+s1}{Uh\PYZhy{}oh! The distance matrices are different}\PY{l+s+s1}{\PYZsq{}}\PY{p}{)}
\end{Verbatim}


    \begin{Verbatim}[commandchars=\\\{\}]
Difference was: 0.000000
Good! The distance matrices are the same

    \end{Verbatim}

    \begin{Verbatim}[commandchars=\\\{\}]
{\color{incolor}In [{\color{incolor}13}]:} \PY{c+c1}{\PYZsh{} Let\PYZsq{}s compare how fast the implementations are}
         \PY{k}{def} \PY{n+nf}{time\PYZus{}function}\PY{p}{(}\PY{n}{f}\PY{p}{,} \PY{o}{*}\PY{n}{args}\PY{p}{)}\PY{p}{:}
             \PY{l+s+sd}{\PYZdq{}\PYZdq{}\PYZdq{}}
         \PY{l+s+sd}{    Call a function f with args and return the time (in seconds) that it took to execute.}
         \PY{l+s+sd}{    \PYZdq{}\PYZdq{}\PYZdq{}}
             \PY{k+kn}{import} \PY{n+nn}{time}
             \PY{n}{tic} \PY{o}{=} \PY{n}{time}\PY{o}{.}\PY{n}{time}\PY{p}{(}\PY{p}{)}
             \PY{n}{f}\PY{p}{(}\PY{o}{*}\PY{n}{args}\PY{p}{)}
             \PY{n}{toc} \PY{o}{=} \PY{n}{time}\PY{o}{.}\PY{n}{time}\PY{p}{(}\PY{p}{)}
             \PY{k}{return} \PY{n}{toc} \PY{o}{\PYZhy{}} \PY{n}{tic}
         
         \PY{n}{two\PYZus{}loop\PYZus{}time} \PY{o}{=} \PY{n}{time\PYZus{}function}\PY{p}{(}\PY{n}{classifier}\PY{o}{.}\PY{n}{compute\PYZus{}distances\PYZus{}two\PYZus{}loops}\PY{p}{,} \PY{n}{X\PYZus{}test}\PY{p}{)}
         \PY{n+nb}{print}\PY{p}{(}\PY{l+s+s1}{\PYZsq{}}\PY{l+s+s1}{Two loop version took }\PY{l+s+si}{\PYZpc{}f}\PY{l+s+s1}{ seconds}\PY{l+s+s1}{\PYZsq{}} \PY{o}{\PYZpc{}} \PY{n}{two\PYZus{}loop\PYZus{}time}\PY{p}{)}
         
         \PY{n}{one\PYZus{}loop\PYZus{}time} \PY{o}{=} \PY{n}{time\PYZus{}function}\PY{p}{(}\PY{n}{classifier}\PY{o}{.}\PY{n}{compute\PYZus{}distances\PYZus{}one\PYZus{}loop}\PY{p}{,} \PY{n}{X\PYZus{}test}\PY{p}{)}
         \PY{n+nb}{print}\PY{p}{(}\PY{l+s+s1}{\PYZsq{}}\PY{l+s+s1}{One loop version took }\PY{l+s+si}{\PYZpc{}f}\PY{l+s+s1}{ seconds}\PY{l+s+s1}{\PYZsq{}} \PY{o}{\PYZpc{}} \PY{n}{one\PYZus{}loop\PYZus{}time}\PY{p}{)}
         
         \PY{n}{no\PYZus{}loop\PYZus{}time} \PY{o}{=} \PY{n}{time\PYZus{}function}\PY{p}{(}\PY{n}{classifier}\PY{o}{.}\PY{n}{compute\PYZus{}distances\PYZus{}no\PYZus{}loops}\PY{p}{,} \PY{n}{X\PYZus{}test}\PY{p}{)}
         \PY{n+nb}{print}\PY{p}{(}\PY{l+s+s1}{\PYZsq{}}\PY{l+s+s1}{No loop version took }\PY{l+s+si}{\PYZpc{}f}\PY{l+s+s1}{ seconds}\PY{l+s+s1}{\PYZsq{}} \PY{o}{\PYZpc{}} \PY{n}{no\PYZus{}loop\PYZus{}time}\PY{p}{)}
         
         \PY{c+c1}{\PYZsh{} you should see significantly faster performance with the fully vectorized implementation}
\end{Verbatim}


    \begin{Verbatim}[commandchars=\\\{\}]
Two loop version took 32.560608 seconds
One loop version took 36.555758 seconds
No loop version took 0.202962 seconds

    \end{Verbatim}

    \subsubsection{Cross-validation}\label{cross-validation}

We have implemented the k-Nearest Neighbor classifier but we set the
value k = 5 arbitrarily. We will now determine the best value of this
hyperparameter with cross-validation.

    \begin{Verbatim}[commandchars=\\\{\}]
{\color{incolor}In [{\color{incolor}14}]:} \PY{n}{num\PYZus{}folds} \PY{o}{=} \PY{l+m+mi}{5}
         \PY{n}{k\PYZus{}choices} \PY{o}{=} \PY{p}{[}\PY{l+m+mi}{1}\PY{p}{,} \PY{l+m+mi}{3}\PY{p}{,} \PY{l+m+mi}{5}\PY{p}{,} \PY{l+m+mi}{8}\PY{p}{,} \PY{l+m+mi}{10}\PY{p}{,} \PY{l+m+mi}{12}\PY{p}{,} \PY{l+m+mi}{15}\PY{p}{,} \PY{l+m+mi}{20}\PY{p}{,} \PY{l+m+mi}{50}\PY{p}{,} \PY{l+m+mi}{100}\PY{p}{]}
         \PY{c+c1}{\PYZsh{}k\PYZus{}choices = [1, 3]}
         \PY{n}{k\PYZus{}to\PYZus{}accuracies} \PY{o}{=} \PY{p}{\PYZob{}}\PY{p}{\PYZcb{}}
         \PY{n}{accuracy} \PY{o}{=} \PY{n}{np}\PY{o}{.}\PY{n}{zeros}\PY{p}{(}\PY{p}{(}\PY{l+m+mi}{10}\PY{p}{,} \PY{l+m+mi}{5}\PY{p}{)}\PY{p}{,} \PY{n}{dtype}\PY{o}{=}\PY{n}{np}\PY{o}{.}\PY{n}{float}\PY{p}{)}
         \PY{n}{X\PYZus{}train\PYZus{}folds} \PY{o}{=} \PY{p}{[}\PY{p}{]}
         \PY{n}{y\PYZus{}train\PYZus{}folds} \PY{o}{=} \PY{p}{[}\PY{p}{]}
         
         \PY{n}{X\PYZus{}train\PYZus{}folds} \PY{o}{=} \PY{n}{np}\PY{o}{.}\PY{n}{array\PYZus{}split}\PY{p}{(}\PY{n}{X\PYZus{}train}\PY{p}{,}\PY{n}{num\PYZus{}folds}\PY{p}{)}
         \PY{n}{y\PYZus{}train\PYZus{}folds} \PY{o}{=} \PY{n}{np}\PY{o}{.}\PY{n}{array\PYZus{}split}\PY{p}{(}\PY{n}{y\PYZus{}train}\PY{p}{,}\PY{n}{num\PYZus{}folds}\PY{p}{)}
         \PY{c+c1}{\PYZsh{}X\PYZus{}train\PYZus{}folds = np.array(X\PYZus{}train\PYZus{}folds)}
         \PY{c+c1}{\PYZsh{}y\PYZus{}train\PYZus{}folds = np.array(y\PYZus{}train\PYZus{}folds)}
         \PY{n}{n} \PY{o}{=} \PY{n}{X\PYZus{}train}\PY{o}{.}\PY{n}{shape}\PY{p}{[}\PY{l+m+mi}{0}\PY{p}{]}\PY{o}{/}\PY{n}{num\PYZus{}folds}
         
         \PY{k}{for} \PY{n}{i} \PY{o+ow}{in} \PY{n+nb}{range}\PY{p}{(}\PY{n+nb}{len}\PY{p}{(}\PY{n}{k\PYZus{}choices}\PY{p}{)}\PY{p}{)}\PY{p}{:}
             \PY{n}{obj} \PY{o}{=} \PY{n}{KNearestNeighbor}\PY{p}{(}\PY{p}{)}
             \PY{k}{for} \PY{n}{j} \PY{o+ow}{in} \PY{n+nb}{range}\PY{p}{(}\PY{n}{num\PYZus{}folds}\PY{p}{)}\PY{p}{:}
                 \PY{n}{temp\PYZus{}train\PYZus{}x} \PY{o}{=} \PY{n}{np}\PY{o}{.}\PY{n}{concatenate}\PY{p}{(}\PY{p}{(}\PY{n}{X\PYZus{}train\PYZus{}folds}\PY{p}{[}\PY{p}{:}\PY{n}{j}\PY{p}{]}\PY{o}{+}\PY{n}{X\PYZus{}train\PYZus{}folds}\PY{p}{[}\PY{n}{j}\PY{o}{+}\PY{l+m+mi}{1}\PY{p}{:}\PY{p}{]}\PY{p}{)}\PY{p}{)}
                 \PY{c+c1}{\PYZsh{}print(temp\PYZus{}train\PYZus{}x.shape)}
                 \PY{n}{temp\PYZus{}train\PYZus{}y} \PY{o}{=} \PY{n}{np}\PY{o}{.}\PY{n}{concatenate}\PY{p}{(}\PY{p}{(}\PY{n}{y\PYZus{}train\PYZus{}folds}\PY{p}{[}\PY{p}{:}\PY{n}{j}\PY{p}{]}\PY{o}{+}\PY{n}{y\PYZus{}train\PYZus{}folds}\PY{p}{[}\PY{n}{j}\PY{o}{+}\PY{l+m+mi}{1}\PY{p}{:}\PY{p}{]}\PY{p}{)}\PY{p}{)}
                 \PY{n}{temp\PYZus{}test\PYZus{}x} \PY{o}{=} \PY{n}{X\PYZus{}train\PYZus{}folds}\PY{p}{[}\PY{n}{j}\PY{p}{]}
                 \PY{n}{temp\PYZus{}test\PYZus{}y} \PY{o}{=} \PY{n}{y\PYZus{}train\PYZus{}folds}\PY{p}{[}\PY{n}{j}\PY{p}{]}
                 \PY{n}{obj}\PY{o}{.}\PY{n}{train}\PY{p}{(}\PY{n}{temp\PYZus{}train\PYZus{}x}\PY{p}{,} \PY{n}{temp\PYZus{}train\PYZus{}y}\PY{p}{)}
                 \PY{n}{dists} \PY{o}{=} \PY{n}{obj}\PY{o}{.}\PY{n}{compute\PYZus{}distances\PYZus{}two\PYZus{}loops}\PY{p}{(}\PY{n}{temp\PYZus{}test\PYZus{}x}\PY{p}{)}
                 \PY{n}{pred} \PY{o}{=} \PY{n}{obj}\PY{o}{.}\PY{n}{predict\PYZus{}labels}\PY{p}{(}\PY{n}{dists}\PY{p}{,} \PY{n}{k}\PY{o}{=}\PY{n}{k\PYZus{}choices}\PY{p}{[}\PY{n}{i}\PY{p}{]}\PY{p}{)}
                 \PY{n}{num\PYZus{}correct} \PY{o}{=} \PY{n}{np}\PY{o}{.}\PY{n}{sum}\PY{p}{(}\PY{n}{pred} \PY{o}{==} \PY{n}{temp\PYZus{}test\PYZus{}y}\PY{p}{)}
                 \PY{n}{accuracy}\PY{p}{[}\PY{n}{i}\PY{p}{]}\PY{p}{[}\PY{n}{j}\PY{p}{]} \PY{o}{=} \PY{n+nb}{float}\PY{p}{(}\PY{n}{num\PYZus{}correct}\PY{p}{)}\PY{o}{/}\PY{n}{n}
                 \PY{n}{k\PYZus{}to\PYZus{}accuracies}\PY{p}{[}\PY{n}{k\PYZus{}choices}\PY{p}{[}\PY{n}{i}\PY{p}{]}\PY{p}{]} \PY{o}{=} \PY{n}{accuracy}\PY{p}{[}\PY{n}{i}\PY{p}{]}
         \PY{c+c1}{\PYZsh{}\PYZsh{}\PYZsh{}\PYZsh{}\PYZsh{}\PYZsh{}\PYZsh{}\PYZsh{}\PYZsh{}\PYZsh{}\PYZsh{}\PYZsh{}\PYZsh{}\PYZsh{}\PYZsh{}\PYZsh{}\PYZsh{}\PYZsh{}\PYZsh{}\PYZsh{}\PYZsh{}\PYZsh{}\PYZsh{}\PYZsh{}\PYZsh{}\PYZsh{}\PYZsh{}\PYZsh{}\PYZsh{}\PYZsh{}\PYZsh{}\PYZsh{}\PYZsh{}\PYZsh{}\PYZsh{}\PYZsh{}\PYZsh{}\PYZsh{}\PYZsh{}\PYZsh{}\PYZsh{}\PYZsh{}\PYZsh{}\PYZsh{}\PYZsh{}\PYZsh{}\PYZsh{}\PYZsh{}\PYZsh{}\PYZsh{}\PYZsh{}\PYZsh{}\PYZsh{}\PYZsh{}\PYZsh{}\PYZsh{}\PYZsh{}\PYZsh{}\PYZsh{}\PYZsh{}\PYZsh{}\PYZsh{}\PYZsh{}\PYZsh{}\PYZsh{}\PYZsh{}\PYZsh{}\PYZsh{}\PYZsh{}\PYZsh{}\PYZsh{}\PYZsh{}\PYZsh{}\PYZsh{}\PYZsh{}\PYZsh{}\PYZsh{}\PYZsh{}\PYZsh{}\PYZsh{}}
         \PY{c+c1}{\PYZsh{} TODO:                                                                        \PYZsh{}}
         \PY{c+c1}{\PYZsh{} Split up the training data into folds. After splitting, X\PYZus{}train\PYZus{}folds and    \PYZsh{}}
         \PY{c+c1}{\PYZsh{} y\PYZus{}train\PYZus{}folds should each be lists of length num\PYZus{}folds, where                \PYZsh{}}
         \PY{c+c1}{\PYZsh{} y\PYZus{}train\PYZus{}folds[i] is the label vector for the points in X\PYZus{}train\PYZus{}folds[i].     \PYZsh{}}
         \PY{c+c1}{\PYZsh{} Hint: Look up the numpy array\PYZus{}split function.                                \PYZsh{}}
         \PY{c+c1}{\PYZsh{}\PYZsh{}\PYZsh{}\PYZsh{}\PYZsh{}\PYZsh{}\PYZsh{}\PYZsh{}\PYZsh{}\PYZsh{}\PYZsh{}\PYZsh{}\PYZsh{}\PYZsh{}\PYZsh{}\PYZsh{}\PYZsh{}\PYZsh{}\PYZsh{}\PYZsh{}\PYZsh{}\PYZsh{}\PYZsh{}\PYZsh{}\PYZsh{}\PYZsh{}\PYZsh{}\PYZsh{}\PYZsh{}\PYZsh{}\PYZsh{}\PYZsh{}\PYZsh{}\PYZsh{}\PYZsh{}\PYZsh{}\PYZsh{}\PYZsh{}\PYZsh{}\PYZsh{}\PYZsh{}\PYZsh{}\PYZsh{}\PYZsh{}\PYZsh{}\PYZsh{}\PYZsh{}\PYZsh{}\PYZsh{}\PYZsh{}\PYZsh{}\PYZsh{}\PYZsh{}\PYZsh{}\PYZsh{}\PYZsh{}\PYZsh{}\PYZsh{}\PYZsh{}\PYZsh{}\PYZsh{}\PYZsh{}\PYZsh{}\PYZsh{}\PYZsh{}\PYZsh{}\PYZsh{}\PYZsh{}\PYZsh{}\PYZsh{}\PYZsh{}\PYZsh{}\PYZsh{}\PYZsh{}\PYZsh{}\PYZsh{}\PYZsh{}\PYZsh{}\PYZsh{}\PYZsh{}}
         \PY{c+c1}{\PYZsh{} Your code}
         \PY{c+c1}{\PYZsh{}\PYZsh{}\PYZsh{}\PYZsh{}\PYZsh{}\PYZsh{}\PYZsh{}\PYZsh{}\PYZsh{}\PYZsh{}\PYZsh{}\PYZsh{}\PYZsh{}\PYZsh{}\PYZsh{}\PYZsh{}\PYZsh{}\PYZsh{}\PYZsh{}\PYZsh{}\PYZsh{}\PYZsh{}\PYZsh{}\PYZsh{}\PYZsh{}\PYZsh{}\PYZsh{}\PYZsh{}\PYZsh{}\PYZsh{}\PYZsh{}\PYZsh{}\PYZsh{}\PYZsh{}\PYZsh{}\PYZsh{}\PYZsh{}\PYZsh{}\PYZsh{}\PYZsh{}\PYZsh{}\PYZsh{}\PYZsh{}\PYZsh{}\PYZsh{}\PYZsh{}\PYZsh{}\PYZsh{}\PYZsh{}\PYZsh{}\PYZsh{}\PYZsh{}\PYZsh{}\PYZsh{}\PYZsh{}\PYZsh{}\PYZsh{}\PYZsh{}\PYZsh{}\PYZsh{}\PYZsh{}\PYZsh{}\PYZsh{}\PYZsh{}\PYZsh{}\PYZsh{}\PYZsh{}\PYZsh{}\PYZsh{}\PYZsh{}\PYZsh{}\PYZsh{}\PYZsh{}\PYZsh{}\PYZsh{}\PYZsh{}\PYZsh{}\PYZsh{}\PYZsh{}\PYZsh{}}
         \PY{c+c1}{\PYZsh{}                                 END OF YOUR CODE                             \PYZsh{}}
         \PY{c+c1}{\PYZsh{}\PYZsh{}\PYZsh{}\PYZsh{}\PYZsh{}\PYZsh{}\PYZsh{}\PYZsh{}\PYZsh{}\PYZsh{}\PYZsh{}\PYZsh{}\PYZsh{}\PYZsh{}\PYZsh{}\PYZsh{}\PYZsh{}\PYZsh{}\PYZsh{}\PYZsh{}\PYZsh{}\PYZsh{}\PYZsh{}\PYZsh{}\PYZsh{}\PYZsh{}\PYZsh{}\PYZsh{}\PYZsh{}\PYZsh{}\PYZsh{}\PYZsh{}\PYZsh{}\PYZsh{}\PYZsh{}\PYZsh{}\PYZsh{}\PYZsh{}\PYZsh{}\PYZsh{}\PYZsh{}\PYZsh{}\PYZsh{}\PYZsh{}\PYZsh{}\PYZsh{}\PYZsh{}\PYZsh{}\PYZsh{}\PYZsh{}\PYZsh{}\PYZsh{}\PYZsh{}\PYZsh{}\PYZsh{}\PYZsh{}\PYZsh{}\PYZsh{}\PYZsh{}\PYZsh{}\PYZsh{}\PYZsh{}\PYZsh{}\PYZsh{}\PYZsh{}\PYZsh{}\PYZsh{}\PYZsh{}\PYZsh{}\PYZsh{}\PYZsh{}\PYZsh{}\PYZsh{}\PYZsh{}\PYZsh{}\PYZsh{}\PYZsh{}\PYZsh{}\PYZsh{}\PYZsh{}}
         
         \PY{c+c1}{\PYZsh{} A dictionary holding the accuracies for different values of k that we find}
         \PY{c+c1}{\PYZsh{} when running cross\PYZhy{}validation. After running cross\PYZhy{}validation,}
         \PY{c+c1}{\PYZsh{} k\PYZus{}to\PYZus{}accuracies[k] should be a list of length num\PYZus{}folds giving the different}
         \PY{c+c1}{\PYZsh{} accuracy values that we found when using that value of k.}
         
         
         
         \PY{c+c1}{\PYZsh{}\PYZsh{}\PYZsh{}\PYZsh{}\PYZsh{}\PYZsh{}\PYZsh{}\PYZsh{}\PYZsh{}\PYZsh{}\PYZsh{}\PYZsh{}\PYZsh{}\PYZsh{}\PYZsh{}\PYZsh{}\PYZsh{}\PYZsh{}\PYZsh{}\PYZsh{}\PYZsh{}\PYZsh{}\PYZsh{}\PYZsh{}\PYZsh{}\PYZsh{}\PYZsh{}\PYZsh{}\PYZsh{}\PYZsh{}\PYZsh{}\PYZsh{}\PYZsh{}\PYZsh{}\PYZsh{}\PYZsh{}\PYZsh{}\PYZsh{}\PYZsh{}\PYZsh{}\PYZsh{}\PYZsh{}\PYZsh{}\PYZsh{}\PYZsh{}\PYZsh{}\PYZsh{}\PYZsh{}\PYZsh{}\PYZsh{}\PYZsh{}\PYZsh{}\PYZsh{}\PYZsh{}\PYZsh{}\PYZsh{}\PYZsh{}\PYZsh{}\PYZsh{}\PYZsh{}\PYZsh{}\PYZsh{}\PYZsh{}\PYZsh{}\PYZsh{}\PYZsh{}\PYZsh{}\PYZsh{}\PYZsh{}\PYZsh{}\PYZsh{}\PYZsh{}\PYZsh{}\PYZsh{}\PYZsh{}\PYZsh{}\PYZsh{}\PYZsh{}\PYZsh{}\PYZsh{}}
         \PY{c+c1}{\PYZsh{} TODO:                                                                        \PYZsh{}}
         \PY{c+c1}{\PYZsh{} Perform k\PYZhy{}fold cross validation to find the best value of k. For each        \PYZsh{}}
         \PY{c+c1}{\PYZsh{} possible value of k, run the k\PYZhy{}nearest\PYZhy{}neighbor algorithm num\PYZus{}folds times,   \PYZsh{}}
         \PY{c+c1}{\PYZsh{} where in each case you use all but one of the folds as training data and the \PYZsh{}}
         \PY{c+c1}{\PYZsh{} last fold as a validation set. Store the accuracies for all fold and all     \PYZsh{}}
         \PY{c+c1}{\PYZsh{} values of k in the k\PYZus{}to\PYZus{}accuracies dictionary.                               \PYZsh{}}
         \PY{c+c1}{\PYZsh{}\PYZsh{}\PYZsh{}\PYZsh{}\PYZsh{}\PYZsh{}\PYZsh{}\PYZsh{}\PYZsh{}\PYZsh{}\PYZsh{}\PYZsh{}\PYZsh{}\PYZsh{}\PYZsh{}\PYZsh{}\PYZsh{}\PYZsh{}\PYZsh{}\PYZsh{}\PYZsh{}\PYZsh{}\PYZsh{}\PYZsh{}\PYZsh{}\PYZsh{}\PYZsh{}\PYZsh{}\PYZsh{}\PYZsh{}\PYZsh{}\PYZsh{}\PYZsh{}\PYZsh{}\PYZsh{}\PYZsh{}\PYZsh{}\PYZsh{}\PYZsh{}\PYZsh{}\PYZsh{}\PYZsh{}\PYZsh{}\PYZsh{}\PYZsh{}\PYZsh{}\PYZsh{}\PYZsh{}\PYZsh{}\PYZsh{}\PYZsh{}\PYZsh{}\PYZsh{}\PYZsh{}\PYZsh{}\PYZsh{}\PYZsh{}\PYZsh{}\PYZsh{}\PYZsh{}\PYZsh{}\PYZsh{}\PYZsh{}\PYZsh{}\PYZsh{}\PYZsh{}\PYZsh{}\PYZsh{}\PYZsh{}\PYZsh{}\PYZsh{}\PYZsh{}\PYZsh{}\PYZsh{}\PYZsh{}\PYZsh{}\PYZsh{}\PYZsh{}\PYZsh{}\PYZsh{}}
         \PY{c+c1}{\PYZsh{} Your code}
         \PY{c+c1}{\PYZsh{}\PYZsh{}\PYZsh{}\PYZsh{}\PYZsh{}\PYZsh{}\PYZsh{}\PYZsh{}\PYZsh{}\PYZsh{}\PYZsh{}\PYZsh{}\PYZsh{}\PYZsh{}\PYZsh{}\PYZsh{}\PYZsh{}\PYZsh{}\PYZsh{}\PYZsh{}\PYZsh{}\PYZsh{}\PYZsh{}\PYZsh{}\PYZsh{}\PYZsh{}\PYZsh{}\PYZsh{}\PYZsh{}\PYZsh{}\PYZsh{}\PYZsh{}\PYZsh{}\PYZsh{}\PYZsh{}\PYZsh{}\PYZsh{}\PYZsh{}\PYZsh{}\PYZsh{}\PYZsh{}\PYZsh{}\PYZsh{}\PYZsh{}\PYZsh{}\PYZsh{}\PYZsh{}\PYZsh{}\PYZsh{}\PYZsh{}\PYZsh{}\PYZsh{}\PYZsh{}\PYZsh{}\PYZsh{}\PYZsh{}\PYZsh{}\PYZsh{}\PYZsh{}\PYZsh{}\PYZsh{}\PYZsh{}\PYZsh{}\PYZsh{}\PYZsh{}\PYZsh{}\PYZsh{}\PYZsh{}\PYZsh{}\PYZsh{}\PYZsh{}\PYZsh{}\PYZsh{}\PYZsh{}\PYZsh{}\PYZsh{}\PYZsh{}\PYZsh{}\PYZsh{}\PYZsh{}}
         \PY{c+c1}{\PYZsh{}                                 END OF YOUR CODE                             \PYZsh{}}
         \PY{c+c1}{\PYZsh{}\PYZsh{}\PYZsh{}\PYZsh{}\PYZsh{}\PYZsh{}\PYZsh{}\PYZsh{}\PYZsh{}\PYZsh{}\PYZsh{}\PYZsh{}\PYZsh{}\PYZsh{}\PYZsh{}\PYZsh{}\PYZsh{}\PYZsh{}\PYZsh{}\PYZsh{}\PYZsh{}\PYZsh{}\PYZsh{}\PYZsh{}\PYZsh{}\PYZsh{}\PYZsh{}\PYZsh{}\PYZsh{}\PYZsh{}\PYZsh{}\PYZsh{}\PYZsh{}\PYZsh{}\PYZsh{}\PYZsh{}\PYZsh{}\PYZsh{}\PYZsh{}\PYZsh{}\PYZsh{}\PYZsh{}\PYZsh{}\PYZsh{}\PYZsh{}\PYZsh{}\PYZsh{}\PYZsh{}\PYZsh{}\PYZsh{}\PYZsh{}\PYZsh{}\PYZsh{}\PYZsh{}\PYZsh{}\PYZsh{}\PYZsh{}\PYZsh{}\PYZsh{}\PYZsh{}\PYZsh{}\PYZsh{}\PYZsh{}\PYZsh{}\PYZsh{}\PYZsh{}\PYZsh{}\PYZsh{}\PYZsh{}\PYZsh{}\PYZsh{}\PYZsh{}\PYZsh{}\PYZsh{}\PYZsh{}\PYZsh{}\PYZsh{}\PYZsh{}\PYZsh{}\PYZsh{}}
         
         \PY{c+c1}{\PYZsh{} Print out the computed accuracies}
         \PY{k}{for} \PY{n}{k} \PY{o+ow}{in} \PY{n}{k\PYZus{}choices}\PY{p}{:}
             \PY{k}{for} \PY{n}{accuracy} \PY{o+ow}{in} \PY{n}{k\PYZus{}to\PYZus{}accuracies}\PY{p}{[}\PY{n}{k}\PY{p}{]}\PY{p}{:}
                 \PY{n+nb}{print}\PY{p}{(}\PY{l+s+s1}{\PYZsq{}}\PY{l+s+s1}{k = }\PY{l+s+si}{\PYZpc{}d}\PY{l+s+s1}{, accuracy = }\PY{l+s+si}{\PYZpc{}f}\PY{l+s+s1}{\PYZsq{}} \PY{o}{\PYZpc{}} \PY{p}{(}\PY{n}{k}\PY{p}{,} \PY{n}{accuracy}\PY{p}{)}\PY{p}{)}
         \PY{c+c1}{\PYZsh{}print(k\PYZus{}to\PYZus{}accuracies, accuracy)}
\end{Verbatim}


    \begin{Verbatim}[commandchars=\\\{\}]
k = 1, accuracy = 0.263000
k = 1, accuracy = 0.257000
k = 1, accuracy = 0.264000
k = 1, accuracy = 0.278000
k = 1, accuracy = 0.266000
k = 3, accuracy = 0.239000
k = 3, accuracy = 0.249000
k = 3, accuracy = 0.240000
k = 3, accuracy = 0.266000
k = 3, accuracy = 0.254000
k = 5, accuracy = 0.248000
k = 5, accuracy = 0.266000
k = 5, accuracy = 0.280000
k = 5, accuracy = 0.292000
k = 5, accuracy = 0.280000
k = 8, accuracy = 0.262000
k = 8, accuracy = 0.282000
k = 8, accuracy = 0.273000
k = 8, accuracy = 0.290000
k = 8, accuracy = 0.273000
k = 10, accuracy = 0.265000
k = 10, accuracy = 0.296000
k = 10, accuracy = 0.276000
k = 10, accuracy = 0.284000
k = 10, accuracy = 0.280000
k = 12, accuracy = 0.260000
k = 12, accuracy = 0.295000
k = 12, accuracy = 0.279000
k = 12, accuracy = 0.283000
k = 12, accuracy = 0.280000
k = 15, accuracy = 0.252000
k = 15, accuracy = 0.289000
k = 15, accuracy = 0.278000
k = 15, accuracy = 0.282000
k = 15, accuracy = 0.274000
k = 20, accuracy = 0.270000
k = 20, accuracy = 0.279000
k = 20, accuracy = 0.279000
k = 20, accuracy = 0.282000
k = 20, accuracy = 0.285000
k = 50, accuracy = 0.271000
k = 50, accuracy = 0.288000
k = 50, accuracy = 0.278000
k = 50, accuracy = 0.269000
k = 50, accuracy = 0.266000
k = 100, accuracy = 0.256000
k = 100, accuracy = 0.270000
k = 100, accuracy = 0.263000
k = 100, accuracy = 0.256000
k = 100, accuracy = 0.263000

    \end{Verbatim}

    \begin{Verbatim}[commandchars=\\\{\}]
{\color{incolor}In [{\color{incolor}15}]:} \PY{c+c1}{\PYZsh{} plot the raw observations}
         \PY{k}{for} \PY{n}{k} \PY{o+ow}{in} \PY{n}{k\PYZus{}choices}\PY{p}{:}
             \PY{n}{accuracies} \PY{o}{=} \PY{n}{k\PYZus{}to\PYZus{}accuracies}\PY{p}{[}\PY{n}{k}\PY{p}{]}
             \PY{n}{plt}\PY{o}{.}\PY{n}{scatter}\PY{p}{(}\PY{p}{[}\PY{n}{k}\PY{p}{]} \PY{o}{*} \PY{n+nb}{len}\PY{p}{(}\PY{n}{accuracies}\PY{p}{)}\PY{p}{,} \PY{n}{accuracies}\PY{p}{)}
         
         \PY{c+c1}{\PYZsh{}plot the trend line with error bars that correspond to standard deviation}
         \PY{n}{accuracies\PYZus{}mean} \PY{o}{=} \PY{n}{np}\PY{o}{.}\PY{n}{array}\PY{p}{(}\PY{p}{[}\PY{n}{np}\PY{o}{.}\PY{n}{mean}\PY{p}{(}\PY{n}{v}\PY{p}{)} \PY{k}{for} \PY{n}{k}\PY{p}{,}\PY{n}{v} \PY{o+ow}{in} \PY{n+nb}{sorted}\PY{p}{(}\PY{n}{k\PYZus{}to\PYZus{}accuracies}\PY{o}{.}\PY{n}{items}\PY{p}{(}\PY{p}{)}\PY{p}{)}\PY{p}{]}\PY{p}{)}
         \PY{n}{accuracies\PYZus{}std} \PY{o}{=} \PY{n}{np}\PY{o}{.}\PY{n}{array}\PY{p}{(}\PY{p}{[}\PY{n}{np}\PY{o}{.}\PY{n}{std}\PY{p}{(}\PY{n}{v}\PY{p}{)} \PY{k}{for} \PY{n}{k}\PY{p}{,}\PY{n}{v} \PY{o+ow}{in} \PY{n+nb}{sorted}\PY{p}{(}\PY{n}{k\PYZus{}to\PYZus{}accuracies}\PY{o}{.}\PY{n}{items}\PY{p}{(}\PY{p}{)}\PY{p}{)}\PY{p}{]}\PY{p}{)}
         \PY{n}{plt}\PY{o}{.}\PY{n}{errorbar}\PY{p}{(}\PY{n}{k\PYZus{}choices}\PY{p}{,} \PY{n}{accuracies\PYZus{}mean}\PY{p}{,} \PY{n}{yerr}\PY{o}{=}\PY{n}{accuracies\PYZus{}std}\PY{p}{)}
         \PY{n}{plt}\PY{o}{.}\PY{n}{title}\PY{p}{(}\PY{l+s+s1}{\PYZsq{}}\PY{l+s+s1}{Cross\PYZhy{}validation on k}\PY{l+s+s1}{\PYZsq{}}\PY{p}{)}
         \PY{n}{plt}\PY{o}{.}\PY{n}{xlabel}\PY{p}{(}\PY{l+s+s1}{\PYZsq{}}\PY{l+s+s1}{k}\PY{l+s+s1}{\PYZsq{}}\PY{p}{)}
         \PY{n}{plt}\PY{o}{.}\PY{n}{ylabel}\PY{p}{(}\PY{l+s+s1}{\PYZsq{}}\PY{l+s+s1}{Cross\PYZhy{}validation accuracy}\PY{l+s+s1}{\PYZsq{}}\PY{p}{)}
         \PY{n}{plt}\PY{o}{.}\PY{n}{show}\PY{p}{(}\PY{p}{)}
\end{Verbatim}


    \begin{Verbatim}[commandchars=\\\{\}]
The history saving thread hit an unexpected error (OperationalError('database is locked',)).History will not be written to the database.

    \end{Verbatim}

    \begin{center}
    \adjustimage{max size={0.9\linewidth}{0.9\paperheight}}{output_22_1.png}
    \end{center}
    { \hspace*{\fill} \\}
    
    \begin{Verbatim}[commandchars=\\\{\}]
{\color{incolor}In [{\color{incolor}27}]:} \PY{c+c1}{\PYZsh{} Based on the cross\PYZhy{}validation results above, choose the best value for k,   }
         \PY{c+c1}{\PYZsh{} retrain the classifier using all the training data, and test it on the test}
         \PY{c+c1}{\PYZsh{} data. You should be able to get above 28\PYZpc{} accuracy on the test data.}
         \PY{n}{best\PYZus{}k} \PY{o}{=} \PY{l+m+mi}{10}
         
         \PY{n}{classifier} \PY{o}{=} \PY{n}{KNearestNeighbor}\PY{p}{(}\PY{p}{)}
         \PY{n}{classifier}\PY{o}{.}\PY{n}{train}\PY{p}{(}\PY{n}{X\PYZus{}train}\PY{p}{,} \PY{n}{y\PYZus{}train}\PY{p}{)}
         \PY{n}{y\PYZus{}test\PYZus{}pred} \PY{o}{=} \PY{n}{classifier}\PY{o}{.}\PY{n}{predict}\PY{p}{(}\PY{n}{X\PYZus{}test}\PY{p}{,} \PY{n}{k}\PY{o}{=}\PY{n}{best\PYZus{}k}\PY{p}{)}
         
         \PY{c+c1}{\PYZsh{} Compute and display the accuracy}
         \PY{n}{num\PYZus{}correct} \PY{o}{=} \PY{n}{np}\PY{o}{.}\PY{n}{sum}\PY{p}{(}\PY{n}{y\PYZus{}test\PYZus{}pred} \PY{o}{==} \PY{n}{y\PYZus{}test}\PY{p}{)}
         \PY{n}{accuracy} \PY{o}{=} \PY{n+nb}{float}\PY{p}{(}\PY{n}{num\PYZus{}correct}\PY{p}{)} \PY{o}{/} \PY{n}{num\PYZus{}test}
         \PY{n+nb}{print}\PY{p}{(}\PY{l+s+s1}{\PYZsq{}}\PY{l+s+s1}{Got }\PY{l+s+si}{\PYZpc{}d}\PY{l+s+s1}{ / }\PY{l+s+si}{\PYZpc{}d}\PY{l+s+s1}{ correct =\PYZgt{} accuracy: }\PY{l+s+si}{\PYZpc{}f}\PY{l+s+s1}{\PYZsq{}} \PY{o}{\PYZpc{}} \PY{p}{(}\PY{n}{num\PYZus{}correct}\PY{p}{,} \PY{n}{num\PYZus{}test}\PY{p}{,} \PY{n}{accuracy}\PY{p}{)}\PY{p}{)}
\end{Verbatim}


    \begin{Verbatim}[commandchars=\\\{\}]
Got 141 / 500 correct => accuracy: 0.282000

    \end{Verbatim}

    \textbf{Inline Question 3} Which of the following statements about
\(k\)-Nearest Neighbor (\(k\)-NN) are true in a classification setting,
and for all \(k\)? Select all that apply. 1. The training error of a
1-NN will always be better than that of 5-NN. 2. The test error of a
1-NN will always be better than that of a 5-NN. 3. The decision boundary
of the k-NN classifier is linear. 4. The time needed to classify a test
example with the k-NN classifier grows with the size of the training
set. 5. None of the above.

\emph{Your Answer}: The training error of 1-NN will always be better
than that of 5-NN; The time needed to classify a test example with the
k-NN classifier grows with the size of the training set

\emph{Your explanation}: Since it uses only 1 closest neighbour, the
error is often low when compared with 5NN. higher the number of values
in the dataset, it takes more time because it has to compare each test
point with with more train data.


    % Add a bibliography block to the postdoc
    
    
    
    \end{document}
